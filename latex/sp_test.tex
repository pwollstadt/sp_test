% Generated by Sphinx.
\def\sphinxdocclass{report}
\documentclass[letterpaper,10pt,english]{sphinxmanual}
\usepackage[utf8]{inputenc}
\DeclareUnicodeCharacter{00A0}{\nobreakspace}
\usepackage{cmap}
\usepackage[T1]{fontenc}

\usepackage{babel}
\usepackage{times}
\usepackage[Bjarne]{fncychap}
\usepackage{longtable}
\usepackage{sphinx}
\usepackage{multirow}
\usepackage{eqparbox}


\addto\captionsenglish{\renewcommand{\figurename}{Fig. }}
\addto\captionsenglish{\renewcommand{\tablename}{Table }}
\SetupFloatingEnvironment{literal-block}{name=Listing }



\title{sp\_test Documentation}
\date{March 15, 2016}
\release{0.1}
\author{PW}
\newcommand{\sphinxlogo}{}
\renewcommand{\releasename}{Release}
\setcounter{tocdepth}{1}
\makeindex

\makeatletter
\def\PYG@reset{\let\PYG@it=\relax \let\PYG@bf=\relax%
    \let\PYG@ul=\relax \let\PYG@tc=\relax%
    \let\PYG@bc=\relax \let\PYG@ff=\relax}
\def\PYG@tok#1{\csname PYG@tok@#1\endcsname}
\def\PYG@toks#1+{\ifx\relax#1\empty\else%
    \PYG@tok{#1}\expandafter\PYG@toks\fi}
\def\PYG@do#1{\PYG@bc{\PYG@tc{\PYG@ul{%
    \PYG@it{\PYG@bf{\PYG@ff{#1}}}}}}}
\def\PYG#1#2{\PYG@reset\PYG@toks#1+\relax+\PYG@do{#2}}

\expandafter\def\csname PYG@tok@c1\endcsname{\let\PYG@it=\textit\def\PYG@tc##1{\textcolor[rgb]{0.25,0.50,0.56}{##1}}}
\expandafter\def\csname PYG@tok@na\endcsname{\def\PYG@tc##1{\textcolor[rgb]{0.25,0.44,0.63}{##1}}}
\expandafter\def\csname PYG@tok@se\endcsname{\let\PYG@bf=\textbf\def\PYG@tc##1{\textcolor[rgb]{0.25,0.44,0.63}{##1}}}
\expandafter\def\csname PYG@tok@nb\endcsname{\def\PYG@tc##1{\textcolor[rgb]{0.00,0.44,0.13}{##1}}}
\expandafter\def\csname PYG@tok@m\endcsname{\def\PYG@tc##1{\textcolor[rgb]{0.13,0.50,0.31}{##1}}}
\expandafter\def\csname PYG@tok@kr\endcsname{\let\PYG@bf=\textbf\def\PYG@tc##1{\textcolor[rgb]{0.00,0.44,0.13}{##1}}}
\expandafter\def\csname PYG@tok@sr\endcsname{\def\PYG@tc##1{\textcolor[rgb]{0.14,0.33,0.53}{##1}}}
\expandafter\def\csname PYG@tok@vc\endcsname{\def\PYG@tc##1{\textcolor[rgb]{0.73,0.38,0.84}{##1}}}
\expandafter\def\csname PYG@tok@ow\endcsname{\let\PYG@bf=\textbf\def\PYG@tc##1{\textcolor[rgb]{0.00,0.44,0.13}{##1}}}
\expandafter\def\csname PYG@tok@sc\endcsname{\def\PYG@tc##1{\textcolor[rgb]{0.25,0.44,0.63}{##1}}}
\expandafter\def\csname PYG@tok@kt\endcsname{\def\PYG@tc##1{\textcolor[rgb]{0.56,0.13,0.00}{##1}}}
\expandafter\def\csname PYG@tok@cp\endcsname{\def\PYG@tc##1{\textcolor[rgb]{0.00,0.44,0.13}{##1}}}
\expandafter\def\csname PYG@tok@o\endcsname{\def\PYG@tc##1{\textcolor[rgb]{0.40,0.40,0.40}{##1}}}
\expandafter\def\csname PYG@tok@ge\endcsname{\let\PYG@it=\textit}
\expandafter\def\csname PYG@tok@cm\endcsname{\let\PYG@it=\textit\def\PYG@tc##1{\textcolor[rgb]{0.25,0.50,0.56}{##1}}}
\expandafter\def\csname PYG@tok@s1\endcsname{\def\PYG@tc##1{\textcolor[rgb]{0.25,0.44,0.63}{##1}}}
\expandafter\def\csname PYG@tok@no\endcsname{\def\PYG@tc##1{\textcolor[rgb]{0.38,0.68,0.84}{##1}}}
\expandafter\def\csname PYG@tok@sh\endcsname{\def\PYG@tc##1{\textcolor[rgb]{0.25,0.44,0.63}{##1}}}
\expandafter\def\csname PYG@tok@vi\endcsname{\def\PYG@tc##1{\textcolor[rgb]{0.73,0.38,0.84}{##1}}}
\expandafter\def\csname PYG@tok@kn\endcsname{\let\PYG@bf=\textbf\def\PYG@tc##1{\textcolor[rgb]{0.00,0.44,0.13}{##1}}}
\expandafter\def\csname PYG@tok@gh\endcsname{\let\PYG@bf=\textbf\def\PYG@tc##1{\textcolor[rgb]{0.00,0.00,0.50}{##1}}}
\expandafter\def\csname PYG@tok@sx\endcsname{\def\PYG@tc##1{\textcolor[rgb]{0.78,0.36,0.04}{##1}}}
\expandafter\def\csname PYG@tok@mf\endcsname{\def\PYG@tc##1{\textcolor[rgb]{0.13,0.50,0.31}{##1}}}
\expandafter\def\csname PYG@tok@si\endcsname{\let\PYG@it=\textit\def\PYG@tc##1{\textcolor[rgb]{0.44,0.63,0.82}{##1}}}
\expandafter\def\csname PYG@tok@w\endcsname{\def\PYG@tc##1{\textcolor[rgb]{0.73,0.73,0.73}{##1}}}
\expandafter\def\csname PYG@tok@cpf\endcsname{\let\PYG@it=\textit\def\PYG@tc##1{\textcolor[rgb]{0.25,0.50,0.56}{##1}}}
\expandafter\def\csname PYG@tok@go\endcsname{\def\PYG@tc##1{\textcolor[rgb]{0.20,0.20,0.20}{##1}}}
\expandafter\def\csname PYG@tok@kc\endcsname{\let\PYG@bf=\textbf\def\PYG@tc##1{\textcolor[rgb]{0.00,0.44,0.13}{##1}}}
\expandafter\def\csname PYG@tok@il\endcsname{\def\PYG@tc##1{\textcolor[rgb]{0.13,0.50,0.31}{##1}}}
\expandafter\def\csname PYG@tok@s2\endcsname{\def\PYG@tc##1{\textcolor[rgb]{0.25,0.44,0.63}{##1}}}
\expandafter\def\csname PYG@tok@nf\endcsname{\def\PYG@tc##1{\textcolor[rgb]{0.02,0.16,0.49}{##1}}}
\expandafter\def\csname PYG@tok@nl\endcsname{\let\PYG@bf=\textbf\def\PYG@tc##1{\textcolor[rgb]{0.00,0.13,0.44}{##1}}}
\expandafter\def\csname PYG@tok@nt\endcsname{\let\PYG@bf=\textbf\def\PYG@tc##1{\textcolor[rgb]{0.02,0.16,0.45}{##1}}}
\expandafter\def\csname PYG@tok@ne\endcsname{\def\PYG@tc##1{\textcolor[rgb]{0.00,0.44,0.13}{##1}}}
\expandafter\def\csname PYG@tok@gs\endcsname{\let\PYG@bf=\textbf}
\expandafter\def\csname PYG@tok@c\endcsname{\let\PYG@it=\textit\def\PYG@tc##1{\textcolor[rgb]{0.25,0.50,0.56}{##1}}}
\expandafter\def\csname PYG@tok@gi\endcsname{\def\PYG@tc##1{\textcolor[rgb]{0.00,0.63,0.00}{##1}}}
\expandafter\def\csname PYG@tok@gd\endcsname{\def\PYG@tc##1{\textcolor[rgb]{0.63,0.00,0.00}{##1}}}
\expandafter\def\csname PYG@tok@ch\endcsname{\let\PYG@it=\textit\def\PYG@tc##1{\textcolor[rgb]{0.25,0.50,0.56}{##1}}}
\expandafter\def\csname PYG@tok@ni\endcsname{\let\PYG@bf=\textbf\def\PYG@tc##1{\textcolor[rgb]{0.84,0.33,0.22}{##1}}}
\expandafter\def\csname PYG@tok@nv\endcsname{\def\PYG@tc##1{\textcolor[rgb]{0.73,0.38,0.84}{##1}}}
\expandafter\def\csname PYG@tok@gt\endcsname{\def\PYG@tc##1{\textcolor[rgb]{0.00,0.27,0.87}{##1}}}
\expandafter\def\csname PYG@tok@sd\endcsname{\let\PYG@it=\textit\def\PYG@tc##1{\textcolor[rgb]{0.25,0.44,0.63}{##1}}}
\expandafter\def\csname PYG@tok@mh\endcsname{\def\PYG@tc##1{\textcolor[rgb]{0.13,0.50,0.31}{##1}}}
\expandafter\def\csname PYG@tok@gp\endcsname{\let\PYG@bf=\textbf\def\PYG@tc##1{\textcolor[rgb]{0.78,0.36,0.04}{##1}}}
\expandafter\def\csname PYG@tok@kd\endcsname{\let\PYG@bf=\textbf\def\PYG@tc##1{\textcolor[rgb]{0.00,0.44,0.13}{##1}}}
\expandafter\def\csname PYG@tok@err\endcsname{\def\PYG@bc##1{\setlength{\fboxsep}{0pt}\fcolorbox[rgb]{1.00,0.00,0.00}{1,1,1}{\strut ##1}}}
\expandafter\def\csname PYG@tok@mi\endcsname{\def\PYG@tc##1{\textcolor[rgb]{0.13,0.50,0.31}{##1}}}
\expandafter\def\csname PYG@tok@nc\endcsname{\let\PYG@bf=\textbf\def\PYG@tc##1{\textcolor[rgb]{0.05,0.52,0.71}{##1}}}
\expandafter\def\csname PYG@tok@mb\endcsname{\def\PYG@tc##1{\textcolor[rgb]{0.13,0.50,0.31}{##1}}}
\expandafter\def\csname PYG@tok@nn\endcsname{\let\PYG@bf=\textbf\def\PYG@tc##1{\textcolor[rgb]{0.05,0.52,0.71}{##1}}}
\expandafter\def\csname PYG@tok@cs\endcsname{\def\PYG@tc##1{\textcolor[rgb]{0.25,0.50,0.56}{##1}}\def\PYG@bc##1{\setlength{\fboxsep}{0pt}\colorbox[rgb]{1.00,0.94,0.94}{\strut ##1}}}
\expandafter\def\csname PYG@tok@mo\endcsname{\def\PYG@tc##1{\textcolor[rgb]{0.13,0.50,0.31}{##1}}}
\expandafter\def\csname PYG@tok@kp\endcsname{\def\PYG@tc##1{\textcolor[rgb]{0.00,0.44,0.13}{##1}}}
\expandafter\def\csname PYG@tok@vg\endcsname{\def\PYG@tc##1{\textcolor[rgb]{0.73,0.38,0.84}{##1}}}
\expandafter\def\csname PYG@tok@ss\endcsname{\def\PYG@tc##1{\textcolor[rgb]{0.32,0.47,0.09}{##1}}}
\expandafter\def\csname PYG@tok@s\endcsname{\def\PYG@tc##1{\textcolor[rgb]{0.25,0.44,0.63}{##1}}}
\expandafter\def\csname PYG@tok@gr\endcsname{\def\PYG@tc##1{\textcolor[rgb]{1.00,0.00,0.00}{##1}}}
\expandafter\def\csname PYG@tok@sb\endcsname{\def\PYG@tc##1{\textcolor[rgb]{0.25,0.44,0.63}{##1}}}
\expandafter\def\csname PYG@tok@gu\endcsname{\let\PYG@bf=\textbf\def\PYG@tc##1{\textcolor[rgb]{0.50,0.00,0.50}{##1}}}
\expandafter\def\csname PYG@tok@nd\endcsname{\let\PYG@bf=\textbf\def\PYG@tc##1{\textcolor[rgb]{0.33,0.33,0.33}{##1}}}
\expandafter\def\csname PYG@tok@bp\endcsname{\def\PYG@tc##1{\textcolor[rgb]{0.00,0.44,0.13}{##1}}}
\expandafter\def\csname PYG@tok@k\endcsname{\let\PYG@bf=\textbf\def\PYG@tc##1{\textcolor[rgb]{0.00,0.44,0.13}{##1}}}

\def\PYGZbs{\char`\\}
\def\PYGZus{\char`\_}
\def\PYGZob{\char`\{}
\def\PYGZcb{\char`\}}
\def\PYGZca{\char`\^}
\def\PYGZam{\char`\&}
\def\PYGZlt{\char`\<}
\def\PYGZgt{\char`\>}
\def\PYGZsh{\char`\#}
\def\PYGZpc{\char`\%}
\def\PYGZdl{\char`\$}
\def\PYGZhy{\char`\-}
\def\PYGZsq{\char`\'}
\def\PYGZdq{\char`\"}
\def\PYGZti{\char`\~}
% for compatibility with earlier versions
\def\PYGZat{@}
\def\PYGZlb{[}
\def\PYGZrb{]}
\makeatother

\renewcommand\PYGZsq{\textquotesingle}

\begin{document}

\maketitle
\tableofcontents
\phantomsection\label{index::doc}


Contents:


\chapter{Autodoc test}
\label{index:welcome-to-sp-test-s-documentation}\label{index:module-multivariate_te}\label{index:autodoc-test}\index{multivariate\_te (module)}
Created on Thu Mar 10 14:24:31 2016

@author: patricia
\index{Multivariate\_te (class in multivariate\_te)}

\begin{fulllineitems}
\phantomsection\label{index:multivariate_te.Multivariate_te}\pysiglinewithargsret{\strong{class }\code{multivariate\_te.}\bfcode{Multivariate\_te}}{\emph{max\_lag}, \emph{min\_lag}, \emph{cmi\_calculator\_name}, \emph{target}, \emph{source\_set=None}}{}
Set up a network analysis using multivariate transfer entropy.

Set parameters necessary for network inference using transfer entropy (TE).
To perform network inference call analyse\_network() on an instance of the
data class.
\begin{quote}\begin{description}
\item[{Parameters}] \leavevmode\begin{itemize}
\item {} 
\textbf{\texttt{max\_lag}} -- maximum number of steps into the past to look for informative
samples (maximum temporal search depth)

\item {} 
\textbf{\texttt{min\_lag}} -- minimum number of steps into the past to look for informative
samples (minimum temporal search depth)

\item {} 
\textbf{\texttt{cmi\_calculator\_name}} -- string with the name of the calculator to be used
for TE estimation

\item {} 
\textbf{\texttt{target}} -- index of the target process

\item {} 
\textbf{\texttt{source\_set}} -- list of process indices used as potential sources (default:
all possible processes, i.e., all processes other than the target
process)

\end{itemize}

\end{description}\end{quote}
\index{analyse\_network (multivariate\_te.Multivariate\_te attribute)}

\begin{fulllineitems}
\phantomsection\label{index:multivariate_te.Multivariate_te.analyse_network}\pysigline{\bfcode{analyse\_network}}
perform network inference on data, has to be run to
first to write results to other attributes

\end{fulllineitems}

\index{conditional\_full (multivariate\_te.Multivariate\_te attribute)}

\begin{fulllineitems}
\phantomsection\label{index:multivariate_te.Multivariate_te.conditional_full}\pysigline{\bfcode{conditional\_full}}
samples in the full conditional set

\end{fulllineitems}

\index{conditional\_sources (multivariate\_te.Multivariate\_te attribute)}

\begin{fulllineitems}
\phantomsection\label{index:multivariate_te.Multivariate_te.conditional_sources}\pysigline{\bfcode{conditional\_sources}}
samples in the conditional set coming from souces

\end{fulllineitems}

\index{conditional\_target (multivariate\_te.Multivariate\_te attribute)}

\begin{fulllineitems}
\phantomsection\label{index:multivariate_te.Multivariate_te.conditional_target}\pysigline{\bfcode{conditional\_target}}
samples in the conditional set coming from target

\end{fulllineitems}

\index{current\_value (multivariate\_te.Multivariate\_te attribute)}

\begin{fulllineitems}
\phantomsection\label{index:multivariate_te.Multivariate_te.current_value}\pysigline{\bfcode{current\_value}}
index of the current value in TE estimation

\end{fulllineitems}

\index{estimator\_name (multivariate\_te.Multivariate\_te attribute)}

\begin{fulllineitems}
\phantomsection\label{index:multivariate_te.Multivariate_te.estimator_name}\pysigline{\bfcode{estimator\_name}}
estimator used for TE estimation

\end{fulllineitems}

\index{max\_lag (multivariate\_te.Multivariate\_te attribute)}

\begin{fulllineitems}
\phantomsection\label{index:multivariate_te.Multivariate_te.max_lag}\pysigline{\bfcode{max\_lag}}
maximum temporal search depth

\end{fulllineitems}

\index{min\_lag (multivariate\_te.Multivariate\_te attribute)}

\begin{fulllineitems}
\phantomsection\label{index:multivariate_te.Multivariate_te.min_lag}\pysigline{\bfcode{min\_lag}}
minimum temporal search depth

\end{fulllineitems}

\index{pvalue\_omnibus (multivariate\_te.Multivariate\_te attribute)}

\begin{fulllineitems}
\phantomsection\label{index:multivariate_te.Multivariate_te.pvalue_omnibus}\pysigline{\bfcode{pvalue\_omnibus}}
p-value of the omnibus test

\end{fulllineitems}

\index{pvalue\_individual\_sources (multivariate\_te.Multivariate\_te attribute)}

\begin{fulllineitems}
\phantomsection\label{index:multivariate_te.Multivariate_te.pvalue_individual_sources}\pysigline{\bfcode{pvalue\_individual\_sources}}
array of p-values for TE from individual
sources to the target

\end{fulllineitems}

\index{sign\_ominbus (multivariate\_te.Multivariate\_te attribute)}

\begin{fulllineitems}
\phantomsection\label{index:multivariate_te.Multivariate_te.sign_ominbus}\pysigline{\bfcode{sign\_ominbus}}
statistical significance of the over-all TE

\end{fulllineitems}

\index{sign\_individual (multivariate\_te.Multivariate\_te attribute)}

\begin{fulllineitems}
\phantomsection\label{index:multivariate_te.Multivariate_te.sign_individual}\pysigline{\bfcode{sign\_individual}}
array of booleans, indicates statistical significance
of TE from individual sources to the target

\end{fulllineitems}

\index{source\_set (multivariate\_te.Multivariate\_te attribute)}

\begin{fulllineitems}
\phantomsection\label{index:multivariate_te.Multivariate_te.source_set}\pysigline{\bfcode{source\_set}}
list with indices of source processes

\end{fulllineitems}

\index{target (multivariate\_te.Multivariate\_te attribute)}

\begin{fulllineitems}
\phantomsection\label{index:multivariate_te.Multivariate_te.target}\pysigline{\bfcode{target}}
index of target process

\end{fulllineitems}

\index{analyse\_network() (multivariate\_te.Multivariate\_te method)}

\begin{fulllineitems}
\pysiglinewithargsret{\bfcode{analyse\_network}}{\emph{data}}{}
Find multivariate transfer entropy between sources and a target.

Find multivariate transfer entropy between all source processes and the
target process. Uses multivariate, non-uniform embedding found through
information maximisation (see Faes, ???, and Lizier, 2012). This is
done in four steps (see Lizier and Faes for details):
(1) find all relevant samples in the target processes' own past, by
\begin{quote}

iteratively adding candidate samples that have significant
conditional mutual information (CMI) with the current value
(conditional on all samples that were added previously)
\end{quote}
\begin{enumerate}
\setcounter{enumi}{1}
\item {} 
find all relevant samples in the source processes' pasts (again
by finding all candidates with significant CMI)

\item {} 
prune the final conditional set by testing the CMI between each
sample in the final set and the current value, conditional on all
other samples in the final set

\item {} 
statistics on the final set of sources (test for over-all transfer
between the final conditional set and the current value, and for
significant transfer of all individual samples in the set)

\end{enumerate}

\end{fulllineitems}


\end{fulllineitems}



\chapter{Indices and tables}
\label{index:indices-and-tables}\begin{itemize}
\item {} 
\DUspan{xref,std,std-ref}{genindex}

\item {} 
\DUspan{xref,std,std-ref}{modindex}

\item {} 
\DUspan{xref,std,std-ref}{search}

\end{itemize}


\renewcommand{\indexname}{Python Module Index}
\begin{theindex}
\def\bigletter#1{{\Large\sffamily#1}\nopagebreak\vspace{1mm}}
\bigletter{m}
\item {\texttt{multivariate\_te}}, \pageref{index:module-multivariate_te}
\end{theindex}

\renewcommand{\indexname}{Index}
\printindex
\end{document}
